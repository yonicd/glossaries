% Options for packages loaded elsewhere
\PassOptionsToPackage{unicode}{hyperref}
\PassOptionsToPackage{hyphens}{url}
%
\documentclass[
]{article}
\usepackage{lmodern}
\usepackage{amsmath}
\usepackage{ifxetex,ifluatex}
\ifnum 0\ifxetex 1\fi\ifluatex 1\fi=0 % if pdftex
  \usepackage[T1]{fontenc}
  \usepackage[utf8]{inputenc}
  \usepackage{textcomp} % provide euro and other symbols
  \usepackage{amssymb}
\else % if luatex or xetex
  \usepackage{unicode-math}
  \defaultfontfeatures{Scale=MatchLowercase}
  \defaultfontfeatures[\rmfamily]{Ligatures=TeX,Scale=1}
\fi
% Use upquote if available, for straight quotes in verbatim environments
\IfFileExists{upquote.sty}{\usepackage{upquote}}{}
\IfFileExists{microtype.sty}{% use microtype if available
  \usepackage[]{microtype}
  \UseMicrotypeSet[protrusion]{basicmath} % disable protrusion for tt fonts
}{}
\makeatletter
\@ifundefined{KOMAClassName}{% if non-KOMA class
  \IfFileExists{parskip.sty}{%
    \usepackage{parskip}
  }{% else
    \setlength{\parindent}{0pt}
    \setlength{\parskip}{6pt plus 2pt minus 1pt}}
}{% if KOMA class
  \KOMAoptions{parskip=half}}
\makeatother
\usepackage{xcolor}
\IfFileExists{xurl.sty}{\usepackage{xurl}}{} % add URL line breaks if available
\IfFileExists{bookmark.sty}{\usepackage{bookmark}}{\usepackage{hyperref}}
\hypersetup{
  pdftitle={Glossary Example},
  pdfauthor={Jonathan Sidi},
  hidelinks,
  pdfcreator={LaTeX via pandoc}}
\urlstyle{same} % disable monospaced font for URLs
\usepackage[margin=1in]{geometry}
\usepackage{color}
\usepackage{fancyvrb}
\newcommand{\VerbBar}{|}
\newcommand{\VERB}{\Verb[commandchars=\\\{\}]}
\DefineVerbatimEnvironment{Highlighting}{Verbatim}{commandchars=\\\{\}}
% Add ',fontsize=\small' for more characters per line
\usepackage{framed}
\definecolor{shadecolor}{RGB}{248,248,248}
\newenvironment{Shaded}{\begin{snugshade}}{\end{snugshade}}
\newcommand{\AlertTok}[1]{\textcolor[rgb]{0.94,0.16,0.16}{#1}}
\newcommand{\AnnotationTok}[1]{\textcolor[rgb]{0.56,0.35,0.01}{\textbf{\textit{#1}}}}
\newcommand{\AttributeTok}[1]{\textcolor[rgb]{0.77,0.63,0.00}{#1}}
\newcommand{\BaseNTok}[1]{\textcolor[rgb]{0.00,0.00,0.81}{#1}}
\newcommand{\BuiltInTok}[1]{#1}
\newcommand{\CharTok}[1]{\textcolor[rgb]{0.31,0.60,0.02}{#1}}
\newcommand{\CommentTok}[1]{\textcolor[rgb]{0.56,0.35,0.01}{\textit{#1}}}
\newcommand{\CommentVarTok}[1]{\textcolor[rgb]{0.56,0.35,0.01}{\textbf{\textit{#1}}}}
\newcommand{\ConstantTok}[1]{\textcolor[rgb]{0.00,0.00,0.00}{#1}}
\newcommand{\ControlFlowTok}[1]{\textcolor[rgb]{0.13,0.29,0.53}{\textbf{#1}}}
\newcommand{\DataTypeTok}[1]{\textcolor[rgb]{0.13,0.29,0.53}{#1}}
\newcommand{\DecValTok}[1]{\textcolor[rgb]{0.00,0.00,0.81}{#1}}
\newcommand{\DocumentationTok}[1]{\textcolor[rgb]{0.56,0.35,0.01}{\textbf{\textit{#1}}}}
\newcommand{\ErrorTok}[1]{\textcolor[rgb]{0.64,0.00,0.00}{\textbf{#1}}}
\newcommand{\ExtensionTok}[1]{#1}
\newcommand{\FloatTok}[1]{\textcolor[rgb]{0.00,0.00,0.81}{#1}}
\newcommand{\FunctionTok}[1]{\textcolor[rgb]{0.00,0.00,0.00}{#1}}
\newcommand{\ImportTok}[1]{#1}
\newcommand{\InformationTok}[1]{\textcolor[rgb]{0.56,0.35,0.01}{\textbf{\textit{#1}}}}
\newcommand{\KeywordTok}[1]{\textcolor[rgb]{0.13,0.29,0.53}{\textbf{#1}}}
\newcommand{\NormalTok}[1]{#1}
\newcommand{\OperatorTok}[1]{\textcolor[rgb]{0.81,0.36,0.00}{\textbf{#1}}}
\newcommand{\OtherTok}[1]{\textcolor[rgb]{0.56,0.35,0.01}{#1}}
\newcommand{\PreprocessorTok}[1]{\textcolor[rgb]{0.56,0.35,0.01}{\textit{#1}}}
\newcommand{\RegionMarkerTok}[1]{#1}
\newcommand{\SpecialCharTok}[1]{\textcolor[rgb]{0.00,0.00,0.00}{#1}}
\newcommand{\SpecialStringTok}[1]{\textcolor[rgb]{0.31,0.60,0.02}{#1}}
\newcommand{\StringTok}[1]{\textcolor[rgb]{0.31,0.60,0.02}{#1}}
\newcommand{\VariableTok}[1]{\textcolor[rgb]{0.00,0.00,0.00}{#1}}
\newcommand{\VerbatimStringTok}[1]{\textcolor[rgb]{0.31,0.60,0.02}{#1}}
\newcommand{\WarningTok}[1]{\textcolor[rgb]{0.56,0.35,0.01}{\textbf{\textit{#1}}}}
\usepackage{longtable,booktabs}
\usepackage{calc} % for calculating minipage widths
% Correct order of tables after \paragraph or \subparagraph
\usepackage{etoolbox}
\makeatletter
\patchcmd\longtable{\par}{\if@noskipsec\mbox{}\fi\par}{}{}
\makeatother
% Allow footnotes in longtable head/foot
\IfFileExists{footnotehyper.sty}{\usepackage{footnotehyper}}{\usepackage{footnote}}
\makesavenoteenv{longtable}
\usepackage{graphicx}
\makeatletter
\def\maxwidth{\ifdim\Gin@nat@width>\linewidth\linewidth\else\Gin@nat@width\fi}
\def\maxheight{\ifdim\Gin@nat@height>\textheight\textheight\else\Gin@nat@height\fi}
\makeatother
% Scale images if necessary, so that they will not overflow the page
% margins by default, and it is still possible to overwrite the defaults
% using explicit options in \includegraphics[width, height, ...]{}
\setkeys{Gin}{width=\maxwidth,height=\maxheight,keepaspectratio}
% Set default figure placement to htbp
\makeatletter
\def\fps@figure{htbp}
\makeatother
\setlength{\emergencystretch}{3em} % prevent overfull lines
\providecommand{\tightlist}{%
  \setlength{\itemsep}{0pt}\setlength{\parskip}{0pt}}
\setcounter{secnumdepth}{-\maxdimen} % remove section numbering
\ifluatex
  \usepackage{selnolig}  % disable illegal ligatures
\fi

\title{Glossary Example}
\author{Jonathan Sidi}
\date{2021-04-23}

\begin{document}
\maketitle

\begin{Shaded}
\begin{Highlighting}[]
\FunctionTok{library}\NormalTok{(glossaries)}
\end{Highlighting}
\end{Shaded}

\begin{Shaded}
\begin{Highlighting}[]
\FunctionTok{add\_definition}\NormalTok{(}
  \AttributeTok{entry =} \StringTok{\textquotesingle{}turtle\textquotesingle{}}\NormalTok{,}
  \AttributeTok{name =} \StringTok{\textquotesingle{}turtle\textquotesingle{}}\NormalTok{,}
  \AttributeTok{desc =} \StringTok{\textquotesingle{}a large marine reptile with a bony or leathery shell and flippers, coming ashore annually on sandy beaches to lay egg\textquotesingle{}}
\NormalTok{  )}
\end{Highlighting}
\end{Shaded}

\begin{Shaded}
\begin{Highlighting}[]
\FunctionTok{add\_acronym}\NormalTok{(}
  \AttributeTok{entry =} \StringTok{\textquotesingle{}aic\textquotesingle{}}\NormalTok{,}
  \AttributeTok{name =} \StringTok{\textquotesingle{}AIC\textquotesingle{}}\NormalTok{,}
  \AttributeTok{desc =} \StringTok{\textquotesingle{}akaike information criterion\textquotesingle{}}
\NormalTok{)}

\FunctionTok{add\_acronym}\NormalTok{(}
  \AttributeTok{entry =} \StringTok{\textquotesingle{}bic\textquotesingle{}}\NormalTok{,}
  \AttributeTok{name =} \StringTok{\textquotesingle{}BIC\textquotesingle{}}\NormalTok{,}
  \AttributeTok{desc =} \StringTok{\textquotesingle{}bayes information criterion\textquotesingle{}}
\NormalTok{)}
\end{Highlighting}
\end{Shaded}

\hypertarget{definitions}{%
\subsection{Definitions}\label{definitions}}

\begin{longtable}[]{@{}ll@{}}
\toprule
\begin{minipage}[b]{(\columnwidth - 1\tabcolsep) * \real{0.06}}\raggedright
Term\strut
\end{minipage} &
\begin{minipage}[b]{(\columnwidth - 1\tabcolsep) * \real{0.94}}\raggedright
\strut
\end{minipage}\tabularnewline
\midrule
\endhead
\begin{minipage}[t]{(\columnwidth - 1\tabcolsep) * \real{0.06}}\raggedright
ocean\strut
\end{minipage} &
\begin{minipage}[t]{(\columnwidth - 1\tabcolsep) * \real{0.94}}\raggedright
oceans are wet from a standpoint of water\strut
\end{minipage}\tabularnewline
\begin{minipage}[t]{(\columnwidth - 1\tabcolsep) * \real{0.06}}\raggedright
turtle\strut
\end{minipage} &
\begin{minipage}[t]{(\columnwidth - 1\tabcolsep) * \real{0.94}}\raggedright
a large marine reptile with a bony or leathery shell and flippers,
coming ashore annually on sandy beaches to lay egg\strut
\end{minipage}\tabularnewline
\bottomrule
\end{longtable}

\hypertarget{acronyms}{%
\subsection{Acronyms}\label{acronyms}}

\begin{longtable}[]{@{}ll@{}}
\toprule
Term &\tabularnewline
\midrule
\endhead
PO & Pacific Ocean\tabularnewline
AIC & akaike information criterion\tabularnewline
BIC & bayes information criterion\tabularnewline
\bottomrule
\end{longtable}

\hypertarget{examples}{%
\subsection{Examples}\label{examples}}

\hypertarget{information-criteria}{%
\subsubsection{Information Criteria}\label{information-criteria}}

In statistics, the \Gls{bic} or Schwarz information criterion (also SIC,
SBC, SBIC) is a criterion for model selection among a finite set of
models; the model with the lowest bayes information criterion (BIC) is
preferred. It is based, in part, on the likelihood function and it is
closely related to the Akaike information criterion (AIC).

When fitting models, it is possible to increase the likelihood by adding
parameters, but doing so may result in overfitting. Both BIC and AIC
attempt to resolve this problem by introducing a penalty term for the
number of parameters in the model; the penalty term is larger in BIC
than in AIC

The BIC was developed by Gideon E. Schwarz and published in a 1978
paper, where he gave a Bayesian argument for adopting it.

\hypertarget{turtles}{%
\subsubsection{Turtles}\label{turtles}}

The Pacific Ocean (PO) is the largest and deepest of Earth's oceanic
divisions. It extends from the Arctic Ocean in the north to the Southern
Ocean (or, depending on definition, to Antarctica) in the south and is
bounded by the continents of Asia and Australia in the west and the
Americas in the east.

Sea turtle live in the Nothern PO one or more life stages, from
hatchling to adult. This includes green, loggerhead, olive ridley,
leatherback, and hawksbill sea turtles---all of which are protected
under the U.S. Endangered Species Act.

\end{document}
